\documentclass{article}


\usepackage[utf8]{inputenc}
\usepackage[T1]{fontenc}
\usepackage{amsmath,amssymb,dsfont}
\numberwithin{equation}{section}
\usepackage{microtype}
\usepackage{graphicx,tikz,pgfplots}
\graphicspath{{images/}}
\pgfplotsset{compat=newest}
\usepackage[hyperref,amsmath,thmmarks]{ntheorem}
\usepackage{aliascnt}
\usepackage[a4paper,centering,bindingoffset=0cm,marginpar=2cm,margin=2.5cm]{geometry}
\usepackage[pagestyles]{titlesec}
\usepackage[font=footnotesize,format=plain,labelfont=sc,textfont=sl,width=0.75\textwidth,labelsep=period]{caption}

\usepackage{bm}
\usepackage{algorithm, algpseudocode}

%'
%' biblatex
%'
\usepackage[backend=biber,maxnames=10,backref=true,hyperref=true,giveninits=true,safeinputenc]{biblatex}
%FINALADDBIB
\addbibresource{journals.bib}%FINALREMOVE
\addbibresource{articles.bib}%FINALREMOVE
\addbibresource{books.bib}%FINALREMOVE
\addbibresource{inproceedings.bib}%FINALREMOVE
\addbibresource{incollection.bib}%FINALREMOVE
\addbibresource{proceedings.bib}%FINALREMOVE
\addbibresource{preprints.bib}%FINALREMOVE
\addbibresource{infmath.bib}%FINALREMOVE
\addbibresource{infmath_books.bib}%FINALREMOVE
\addbibresource{infmath_report.bib}%FINALREMOVE

\DefineBibliographyStrings{english}{%
	backrefpage = {cited on page},
	backrefpages = {cited on pages},
}

\usepackage[pdftex,colorlinks=true,linkcolor=blue,citecolor=green,urlcolor=blue,bookmarks=true,bookmarksnumbered=true]{hyperref}
\hypersetup
{
    pdfauthor={F. Parzer},
    pdftitle={CGN - Mathematical documentation},
    pdfkeywords={Gauss-Newton, nonlinear least-squares, constraints, regularization}
}
\def\sectionautorefname{Section}
\def\subsectionautorefname{Section}

%'
%' defines the pagestyle (ADAPT)
%'
\newpagestyle{headers}
{
	\headrule
	\sethead[\footnotesize\thepage][\footnotesize\sc F. Parzer][]{}{\footnotesize\sc GGN --- Mathematical documentation}{\footnotesize\thepage}
	\setfoot{}{}{}
}
\pagestyle{headers}

%'
%' default paragraph layout.
%'
\postdisplaypenalty= 1000
\widowpenalty = 1000
\clubpenalty = 1000
\displaywidowpenalty = 1000
\setlength{\parindent}{0pt}
\setlength{\parskip}{1ex}
\renewcommand{\labelenumi}{\textit{(\roman{enumi})}}
\renewcommand{\theenumi}{\textit{(\roman{enumi})}}

\newtheorem{lemma}{Lemma}[section]
\def\lemmaautorefname{Lemma}

\newaliascnt{proposition}{lemma}
\newtheorem{proposition}[proposition]{Proposition}
\aliascntresetthe{proposition}
\def\propositionautorefname{Proposition}

\newaliascnt{corollary}{lemma}
\newtheorem{corollary}[corollary]{Corollary}
\aliascntresetthe{corollary}
\def\corollaryautorefname{Corollary}

\newaliascnt{theorem}{lemma}
\newtheorem{theorem}[theorem]{Theorem}
\aliascntresetthe{theorem}
\def\theoremautorefname{Theorem}

\newaliascnt{remark}{lemma}
\newtheorem{remark}[remark]{Remark}
\aliascntresetthe{remark}
\def\theoremautorefname{Remark}

\theorembodyfont{\normalfont}
\newaliascnt{definition}{lemma}
\newtheorem{definition}[definition]{Definition}
\aliascntresetthe{definition}
\def\definitionautorefname{Definition}

\newaliascnt{assumption}{lemma}
\newtheorem{assumption}[assumption]{Assumption}
\aliascntresetthe{assumption}
\def\assumptionautorefname{Assumption}


\theoremstyle{nonumberplain}
\theoremseparator{:}
\theoremheaderfont{\normalfont\itshape}

\theoremsymbol{\ensuremath{\square}}
\newtheorem{proof}{Proof}

\usepackage{bbm}
\usepackage{verbatim}


\newcommand{\fun}{ F }
\newcommand{\funk}{ F_k }
\newcommand{\jac}{ J }
\newcommand{\jack}{ J_k }

\newcommand{\cgn}{{\texttt{CGN}}~}



% Imports my custom definitions


\newcommand{\N}{\mathds{N}}
\newcommand{\Z}{\mathds{Z}}
\newcommand{\R}{\mathds{R}}
\newcommand{\C}{\mathds{C}}

% real part
\let\RE\Re
\let\Re=\undefined
\DeclareMathOperator{\Re}{\RE e}
% imaginary part
\let\IM\Im
\let\Im=\undefined
\DeclareMathOperator{\Im}{\IM m}
% support
\DeclareMathOperator{\supp}{supp}
% sign
\DeclareMathOperator{\sign}{sign}
% absolute Value
\newcommand{\abs}[1]{\left|#1\right|}
% norm
\newcommand{\norm}[1]{\left\|#1\right\|}
% set
\newcommand{\set}[1]{\left\{#1\right\}}
% inner Product
\newcommand{\inner}[2]{\left<#1,#2\right>}
% euler number
\newcommand{\e}{\mathrm e}
% imaginary unit (command \i is stored in \ii)
\let\ii\i
\renewcommand{\i}{\mathrm i}
% derivative
\renewcommand{\d}{\,\mathrm d}

%'
%' comments
%'
\newcommand{\commentO}[1]{\textcolor{red}{#1}}


% Fabian's abbreviations


\DeclareMathOperator*{\esssup}{ess\,sup}
\DeclareMathOperator{\dist}{dist}
\DeclareMathOperator*{\argmin}{argmin}
\DeclareMathOperator*{\argmax}{argmax}

\newcommand{\1}{\mathbbm{1}}
\newcommand{\Set}[2]{\left\{\,#1:\, #2\,\right\}}
\newcommand{\dom}[1]{ \mathcal{D}(#1) }%domain of an unbounded operator
\newcommand{\ran}[1]{ \mathcal{R}(#1) }%range of a linear operator
\newcommand{\nullspace}[1]{ \mathcal{N}(#1) }%the null space of an operator

\newcommand{\linspan}{\mathrm{span}}
\newcommand{\powerset}{\mathfrak{P}}

% spaces
\newcommand{\Rplus}{\R^+}
\newcommand{\EE}{\mathbb{E}}
\newcommand{\HH}{\mathbb{H}}
\newcommand{\UU}{\mathbb{U}}
\newcommand{\VV}{\mathbb{V}}
\newcommand{\WW}{\mathbb{W}}
\newcommand{\XX}{\mathbb{X}}
\newcommand{\YY}{\mathbb{Y}}
\newcommand{\Lin}{\mathrm{Lin}} %the space of linear functions
\newcommand{\BL}{\mathcal{L}} % the space of bounded linear operators

% operators\\
\newcommand{\D}{\,\mathrm D}
\newcommand{\tr}{\mathrm{tr}}
\newcommand{\indicator}[1]{ {\mathbbm{1}_{#1}} }
\newcommand{\Id}{\mathrm{I}}
\newcommand{\Idmat}{\mathbb{I}}
\newcommand{\restr}{\Bigr|}%restriction

% relations
\newcommand{\lx}{\overset{\times}{\leq}}
\newcommand{\gx}{\overset{\times}{\geq}}

% names
\newcommand{\loc}{ \mathrm{loc} }
\newcommand{\rank}{\mathrm{rank}}

% probability theory
\newcommand{\borel}{\mathcal{B}}
\newcommand{\law}{\mathrm{law}}
\newcommand{\Exp}{\mathbb{E}}
\newcommand{\Var}[1]{ \mathrm{var} \left( #1 \right) }
\newcommand{\Cov}{ \mathrm{Cov} }
\newcommand{\given}{ \, \mid \, }
\newcommand{\Condexp}[1]{\mathbb{E}^{#1}}
\newcommand{\deq}{ \overset{d}{=} }
\newcommand{\iidsim}{\overset{\mathrm{iid}}{\sim}}
\newcommand{\normal}{\mathcal{N}}
\newcommand{\laplace}{\mathrm{Lap}}
\newcommand{\PP}{\mathbb{P}}
\newcommand{\QQ}{\mathbb{Q}}
\newcommand{\Prob}{\mathcal{P}}
\newcommand{\Meas}{\mathcal{M}}

%make multiline boxed environment
\usepackage{environ}
\usepackage[ntheorem]{empheq}

\NewEnviron{mybox}[1]{%
	\begin{empheq}[box=\fbox]{#1}
	\BODY
	\end{empheq}
}

% Fabian's comments

\NewEnviron{fabianComment}{%
\color{blue}
(\BODY)
}

\graphicspath{{src/}}

\title{CGN: A Gauss-Newton method for solving penalized nonlinear least-squares problems with linear constraints.}
\author{F. Parzer}

\begin{document}

\maketitle

\begin{abstract}

\end{abstract}



\section{Introduction}

We implement a Gauss-Newton method that solves equality- and inequality constrained nonlinear least-squares problems with multiple regularization terms. That is, the method is applicable to all problems which can be written as
\begin{equation}\tag{CNLS}\label{eq:cnls}
\begin{aligned}
\min_{x \in \R^n} \quad & \frac{1}{2}\norm{F(x_1,\ldots,x_l)}_2^2 + \frac{1}{2} \left( \sum_{i=1}^l \beta_i \norm{P_i(x_i-\bar x_i)}_2^2 \right),\\
\text{s. t. } \quad & Ax = b, \\
& C x \geq d,
& x_i \in \R^{n_i}, \quad n = \sum_i n_i.
\end{aligned}
\end{equation}
where we call the term $ \norm{F(x)}_2^2$ the \textbf{misfit term}, and the terms $ \norm{P_i(x_i-\bar x_i)}_{p_i}^{p_i}$ the \textbf{regularization terms}, and
\begin{itemize}
\item $F: \R^n \to \R^{m_1}$ is a continuously differentiable \textbf{misfit function}. As an example, in nonlinear regression we typically have
\begin{align*}
\fun(x) = \Gamma^{-1/2}(\mathcal G(x) - y),
\end{align*}
where $\mathcal G$ is a nonlinear observation operator,$y$ a given measurement, and $\Gamma \in \R^{m_1\times m_1}$ the covariance matrix of the measurement noise;
\item $\beta > 0$ is a tunable regularization parameter that determines how strongly the regularization terms are weighted with respect to the data term. There are iterative strategies of choosing $\beta$ in case the user does not want to choose it manually; 
\item $P_i \in \R^{r_i \times n_i}$ are weighting matrices of rank $n_i$. We assume throughout that $r_i \geq n_i$ and that $P_i$ has full column rank, which implies that a left-inverse $P_i^+$ of $P_i$ is avaiable, i.e. that
\begin{align*}
P_i^+ P_i = \Idmat_{n_i}.
\end{align*}
In the case of MAP estimation with Gaussian priors ($p_i=2$), $(P_i P_i^\top)^{-1}$ corresponds to the \textbf{prior covariance}. We allow the case where $r_i \neq n_i$, since it is relevant for compressed-sensing applications, where $P_i$ will map into a large redundant dictionary, and we have $r_i \gg n_i$.
\item $\bar x_i \in \R^{n_i}$ is an initial guess which regularizes our fit. In the case of MAP estimation, it corresponds to the \textbf{prior mean};
\item $A \in \R^{c_1 \times n}$ and $b \in \R^{c_1}$ define equality constraints.
\item $C \in \R^{c_2 \times n}$ and $d \in \R^{c_2}$ define the inequality constraints.
\item We will assume throughout that the problem is feasible, i.e. the set $\Set{ x \in \R^n }{ Ax = b, Cx \geq d}$ is nonempty.
\end{itemize}

The \cgn package is a library for solving problem \eqref{eq:cnls} using a generalized version of the Gauss-Newton method.


\section{Main loop}

\subsection{Getting rid of the equality constraints}

First, we explain how the assumption that $A$ has full rank means that we can assume wlog that no equality constraints are present.

Since $A$ is by assumption of full rank, we can compute its QR-decomposition
\begin{align*}
& A^\top = \begin{bmatrix} Q & S \end{bmatrix} \begin{bmatrix}
R \\ 0
\end{bmatrix},\\
& \text{where } Q \in \R^{n \times c_1}, \quad S \in \R^{n \times (n - c_1)},
\end{align*}
and where $[Q, S]\in \R^{n\times n}$ is an orthogonal matrix and $R \in \R^{c_1 \times c_1}$ is an invertible upper triangular matrix.
Then, we define a new variable $q \in \R^n$ by 
\begin{align*}
\begin{bmatrix}
q \\ s
\end{bmatrix} = [Q, S]^\top x = \begin{bmatrix}
Q^\top x \\ S^\top x
\end{bmatrix}, \quad q \in \R^{c_1}, s \in \R^{n-c_1}.
\end{align*}
Then, the constraint $Ax=b$ becomes
\begin{align*}
& \begin{bmatrix}
R^\top & 0
\end{bmatrix} \begin{bmatrix}
Q^\top \\ S^\top
\end{bmatrix} x = b, \\
\Leftrightarrow \quad & R^\top Q^\top = b \\
\Leftrightarrow \quad & R^\top q = b.
\end{align*}
Consequently $q^*$ is determined by $R^\top q^* = b$.

We then reduce the minimization with respect to $x$ to the free variable $s$ through the identity
\begin{align*}
x = [Q, S] [Q, S]^\top x = [Q, S] \begin{bmatrix}
q \\ s
\end{bmatrix} = Q q + S s.
\end{align*}
The fully constrained least-squares problem \eqref{eq:cnls} is consequently equivalent to the inequality-constrained problem
\begin{equation}
\begin{aligned}
\min_{s \in \R^{n-c_1}} \quad & \frac{1}{2} \norm{\tilde F(s)}^2 + \frac{\beta}{2} \norm{\tilde P s - \tilde s}^2, \\
& \tilde C s \geq \tilde d,\\
\text{where } \quad & \tilde F(s)=F(x_1^* + S s), \quad \tilde P = P S, &  \tilde s = P(\bar x - x_1^*), \\
& \tilde C = C S, \quad \tilde d = d - C x_1^*. \\
\end{aligned}
\end{equation}
If $s^*$ is a minimizer of the reduced problem, then
\begin{align*}
& x^* = x_1^* + S s^*
\end{align*}
is a minimizer for \eqref{eq:cnls}.

\subsection{Main loop}

By the preceding section, we can consider without loss of generality the reduced problem
\begin{equation}\tag{CNLS2}\label{eq:cnls2}
\begin{aligned}
\min_{x \in \R^n} \quad & \frac{1}{2}\norm{F(x)}_2^2 + \frac{1}{2} \norm{P x - \tilde x}_2^2,\\
\text{s. t.} \quad & C x \geq d.
\end{aligned}
\end{equation}

The main idea to solve this problem is consecutive linearization. Given the current iterate $x^k$, we define
\begin{align*}
\funk & := F(x^k), \\
\jack & := F'(x^k),
\end{align*}
and solve the linearized problem
\begin{equation}\tag{CLS}\label{eq:cls}
\begin{aligned}
\Delta x^{k+1} = \argmin_{\Delta x \in \R^n} \quad & \frac{1}{2}\norm{\funk + \jack \Delta x}_2^2 + \frac{1}{2} \norm{P \Delta x + P x^k - \tilde x}_2^2,\\
\text{s. t. } \quad & C \Delta x \geq d - C x^k.
\end{aligned}
\end{equation}

Then we define a search path
\begin{align*}
x^{k+1}(h) := x^k + h \Delta x^{k+1}, \qquad h > 0
\end{align*}
and choose a step size $h^k$ using a line search. The next iterate is then
\begin{align*}
x^{k+1} := x^{k+1}(h^k),
\end{align*}
and the procedure is repeated.

\subsection{Solving \eqref{eq:cls}}

Most of this section is based on the classical reference on least-squares problems by Lawson and Hanson \cite{LawHan74}.

It remains to solve the inequality-constrained least-squares problem \eqref{eq:cls}. It is easy to check that \eqref{eq:cls} can be brought into the form
\begin{equation}\tag{LSI}\label{eq:lsi}
\begin{aligned}
\min_x \quad & \frac{1}{2} \norm{ Gx - h }_2^2 \\
\text{s. t. } \quad & C x \geq f,
\end{aligned}
\end{equation}
by setting
\begin{align*}
G &= \begin{bmatrix}
\jack \\ P
\end{bmatrix}, \quad h = \begin{bmatrix}
- \funk \\ \tilde x - P x^k
\end{bmatrix}, \quad f = d - C x^k.
\end{align*}
This problem is an \textbf{inequality-constrained linear least-squares problem (LSI)}, for which a solution can be obtained by solving an associated \textbf{nonnegative least-squares problem}. We describe next how this is done.

\subsubsection{From LSI to LDP}

Since $P$ has full column rank by assumption, so does $G$. Hence, $G$ admits a QR decomposition of the form
\begin{align*}
G = \begin{bmatrix}
U_1 & U_2
\end{bmatrix} \begin{bmatrix} T \\ 0 \end{bmatrix},
\end{align*}
where $T$ is a square upper triangular matrix of full rank.
If we plug this decomposition in the objective function of problem \eqref{eq:lsi}, we have
\begin{align*}
\norm{G x - h}_2^2 & = \norm{ U^\top(Gx-h) }_2^2 \\
& = \norm{ T x - U_1^\top h }_2^2 + \norm{U_2^\top h}_2^2.
\end{align*}
The second term is constant and can therefore be ignored in the optimization. Making the change of variables
\begin{align*}
u = T x - U_1^\top h \quad \Longleftrightarrow \quad x = T^{-1}(u + U_1^\top h).
\end{align*}
problem \eqref{eq:lsi} is now equivalent to
\begin{equation}\tag{LDP}\label{eq:ldp}
\begin{aligned}
\min_u  \quad & \frac{1}{2} \norm{ u }_2^2 \notag\\
\text{s. t.} \quad  & K u \geq l
\\ \text{where} \quad & K = C T^{-1}, \quad l =  d  - K U_1^\top h. \notag
\end{aligned}
\end{equation}
This is a \textbf{least distance problem (LDP)}, for which a solution can be obtained by solving an associated nonnegative least-squares problem (see next section). The solution $x^*$ of problem \eqref{eq:lsi} can then be computed by reverting the change of variables. That is, if $u^*$ solves \eqref{eq:ldp}, then
\begin{align*}
x^* = T^{-1}(u^* + U_1^\top h)
\end{align*}
solves \eqref{eq:lsi}.

\subsubsection{From LDP to NNLS}

Finally, it can be shown (see \cite[chapter 23.4]{LawHan74}) that the LDP problem
\begin{align*}
\min_u \quad &  \frac{1}{2} \norm{u}_2^2 \notag\\
 & Ku \geq l
\end{align*}
is equivalent to the following \textbf{nonnegative least-squares problem (NNLS)}:
\begin{equation}\tag{NNLS}\label{eq:nnls}
\begin{aligned}
\min_v \quad & \frac{1}{2} \norm{Mv - e_{n+1}}_2^2\\
\text{s.t.} \quad & v \geq 0, \\
& \text{where } M = \begin{bmatrix} K^\top \\ l^\top \end{bmatrix}, \text{ and } e_{n+1} = \begin{bmatrix} 0 \\ \vdots \\ 0 \\ 1 \end{bmatrix} \in \R^{n+1}.
\end{aligned}
\end{equation}
If $v_*$ solves this problem, then the solution $u_*$ of \eqref{eq:ldp} is given by
\begin{align*}
u_i^* = - \frac{r_i^*}{r_{n+1}^*}, \quad i=1,\ldots,n,
\end{align*}
where $r^* = M v^*-e_{n+1}$ is the residual of problem \eqref{eq:nnls}.

An algorithm for solving \eqref{eq:nnls} is provided in \cite{LawHan87}. At the moment, we use the implementation of the Lawson-Hanson algorithm given by \texttt{scipy.optimize.nnls}. In the future, we might substitute this with the faster \textbf{FNNLS} algorithm by Bro and de Jong \cite{BroJon97}. However, since the time of computing the QR decompositions usually outweighs the solution of the NNLS problem, this is not a priority.

\subsection{Line search with the Wächter-Biegler filter}

Due to the presence of constraints, we cannot use the same line search strategy for the generalized Gauss-Newton method as in the unconstrained case. Instead, we will use the line search filter by Wächter and Biegler. We provide a rough outline and pseudocode of the method, and refer to the original paper \cite{WaeBie03} for further details.

The idea of a line search filter is to start with a stepsize $h_0 > 0$ and then decrease the steplenght until there is either sufficient decrease in the objective function $\phi$ or in an infeasibility measure, which quantifies how well the proposed iterate satisfies the constraints.

For our case of mixed equality and inequality constraints, we define the infeasibility measure
\begin{align*}
\theta(x) = \norm{ Ax-b }_1 + \abs{ Cx-d }^-,
\end{align*}
where 
\begin{align*}
\abs{z}^- = \sum_{i=1}^n \max(0, -z_i), \qquad \text{for } z \in \R^n.
\end{align*}
A new step $w_{k+1}(h) = w_k + h\Delta w_k$ is then accepted if 
\begin{align*}
& \theta(w_{k+1}(h)) \leq (1-\gamma_\theta) \theta(w_k), \\
\text{or} \quad & \phi(w_{k+1}(h)) \leq \phi(w_k) - \gamma_\phi \theta(w_k),
\end{align*}
where $\phi$ is the least-squares cost function defined in \autoref{sec:gn}, while $\gamma_\theta \in (0,1)$ and $\gamma_\phi \in (0,1)$ are small constants which determine what constitutes "sufficient decrease".

Furthermore, there is a switching condition which ensures that the algorithm still reduces the objective function. A stepsize $h$ meets the switching condition if
\begin{align}
h m_k < 0 \quad \text{and} \quad (-h m_k)^{s_\varphi} h^{1-s_\varphi} > \delta \theta(w_k)^{s_\theta}, \label{eq:switching}
\end{align}
where $m_k =\phi'(w_k)\Delta w_k$ and $\delta > 0$, $s_\theta > 1$ and $s_\varphi > 2 s_\theta$ are tunable parameters. If the switching condition is satisfied, we enforce sufficient decrease in the objective function by accepting the step if and only if
\begin{align}
\phi(w_{k+1}(h)) \leq \phi(w_k) + \eta m_k(h), \label{eq:armijo}
\end{align}
where $\eta \in (0,\frac{1}{2})$ is a small constant.

Finally, we have to keep track of a name-giving filter, which ensures that there are no cycles in the trajectory of our optimization method. A new step is only accepted if it satisfies
\begin{align*}
& \theta(w_{k+1}(h)) < (1-\gamma_\theta)\theta(w), \\
\text{or} \quad & \phi(w_{k+1}(h)) < \phi(w) - \gamma_\phi \theta(w),
\end{align*}
for all iterates $w \in \mathcal F$ in the filter. The new iterate $w_{k+1}$ is then added to the filter if it did not satisfy the switching condition or the Armijo condition.

If no acceptable steplength can be found, a feasibility restoration phase is invoked which computes a new iterate $w_{k+1}$ that is close to the last iterate $w_k$ but also satisfies all constraints. This happens if the stepsize $h$ is decreased below the value
\begin{align*}
h_k^\mathrm{min} := \gamma_h \cdot \begin{cases}
\min \lbrace \gamma_\theta, \frac{\gamma_\phi \theta(w_k)}{- m_k}, \frac{\delta \theta(w_k)^{s_\theta}}{(-m_k)^{s_\phi}} \rbrace, & \text{if } m_k < 0, \\
\gamma_\theta, & \text{otherwise}.
\end{cases}
\end{align*}

\begin{algorithm}[H]
\caption{\texttt{linesearchFilter}}
Given $w_k$, $\Delta w_k$, the current filter $\mathcal F$, and tunable constants $c \in (0,1)$, $\gamma_\theta, \gamma_\phi \in (0,1)$, $\delta > 0$, $s_\theta > 1$, $s_\varphi > 2 s_\theta$, $\eta_\phi \in (0,\frac{1}{2})$ and $\gamma_h \in (0,1)$.
\begin{algorithmic}[1]\label{alg:cgn}
\State $h = 1$;
\State $m_k = \phi'(w_k)\Delta w_k$;
\State $h^\mathrm{min} = \gamma_h \cdot \begin{cases}
\min \lbrace \gamma_\theta, \frac{\gamma_\phi \theta(w_k)}{- m_k}, \frac{\delta \theta(w_k)^{s_\theta}}{(-m_k)^{s_\phi}} \rbrace, & \text{if } m_k < 0, \\
\gamma_\theta, & \text{otherwise}.
\end{cases}$
\Repeat
\State $w_{k+1} = w_k + h \Delta w_k$;
%check switching condition
\If{ $h m_k < 0$ and $(-h m_k)^{s_\varphi} h^{1-s_\varphi} > \delta \theta(w_k)^{s_\theta}$}
%check Armijo condition
\If{ $\phi(w_{k+1}) \leq \phi(w_k) + \eta_\phi h m_k$ }
\If{ $\theta(w_{k+1}) \leq (1-\gamma_\theta) \theta(w)$ or $\phi(w_{k+1}) \leq \phi(w) - \gamma_\phi \theta(w)$ for all $w \in \mathcal F$}
\State break;
\EndIf
\EndIf
\ElsIf{ $\theta(w_{k+1}) \leq (1-\gamma_\theta) \theta(w_k)$ or $\phi(w_{k+1}) \leq \phi(w_k) - \gamma_\phi \theta(w_k)$}
\If{ $\theta(w_{k+1}) \leq (1-\gamma_\theta) \theta(w)$ or $\phi(w_{k+1}) \leq \phi(w) - \gamma_\phi \theta(w)$ for all $w \in \mathcal F$}
% If Armijo does not hold, add step to filter
\If{ $\phi(w_{k+1}) > \phi(w_k) + \eta_\phi h m_k$ }
\State add $w_{k+1}$ to $\mathcal F$;
\EndIf
\State break;
\EndIf
\EndIf

\State $h = c \cdot h$;
\Until {$h \leq h^\mathrm{min}$}
\If {$h \leq h^\mathrm{min}$}
\State find $w_{k+1}$ using feasibility restoration;
\State add $w_k$ to $\mathcal F$;
\EndIf
\State return $w_{k+1}$ and $\mathcal F$;
\end{algorithmic}
\end{algorithm}


\addcontentsline{toc}{section}{Bibliography}

\printbibliography

\end{document}
